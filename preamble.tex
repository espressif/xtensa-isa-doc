% Package includes

\usepackage{caption}
\usepackage{changepage}
\usepackage[style=iso]{datetime2}
\usepackage{enumitem}
\usepackage{graphicx}
\usepackage[a4paper,margin=20mm]{geometry}
\usepackage{float}
\usepackage[colorlinks,citecolor=Navy,linkcolor=Navy]{hyperref}
\usepackage{longtable}
\usepackage{textcomp}
\usepackage[svgnames]{xcolor}
\usepackage[nocolor]{drawstack}

\setlength{\parskip}{0.5\baselineskip}

\setlength{\tabcolsep}{4pt}

\newenvironment{smalltables}{\small}{\normalsize}

% include this to debug page layout:
% \usepackage{showframe}

\captionsetup[table]{labelformat=empty}
\graphicspath{ {./images/} }

% commands to format instruction and register names
\newcommand{\insn}[1]{\texttt{#1}}
\newcommand{\reg}[1]{\texttt{#1}}
\newcommand{\an}{\texttt{a\textit{n}}}

% some formatting helpers
\newcommand{\fixme}[1]{\textcolor{purple}{\textit{FIXME: #1}}}
\newcommand{\note}[1]{
    \begin{adjustwidth}{3cm}{}
    \begin{flushright}
    \textit{#1}
    \end{flushright}
    \vspace{\parskip}
    \end{adjustwidth}
}

% Similar to \cellpt command from drawstack package, but points
% to the next divider between cells. See drawstack.sty for the
% definition of \cellptr. This differs only in "\value{cellnb}-0.5"
% instead of "\value{cellnb}-1".
\newcommand{\cellptrdiv}[1]{
  \draw[<-,line width=0.7pt] (0,\value{cellnb}-0.5) +(2,\value{ptrnb}*0.1) -- +(2.5,\value{ptrnb}*0.45);
  \draw (2.5,\value{ptrnb}*0.5+\value{cellnb}-0.5) node[anchor=west] {#1};
  \addtocounter{ptrnb}{1}
}
