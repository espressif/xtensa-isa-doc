\chapter{Overview of Xtensa Instruction Set Architecture}

\section{Basic facts about Xtensa ISA}

\note{The content of this section is based on \cite[Chapter 3]{leibson2006designing}.}

Xtensa is a post-RISC ISA i.e it derives most of its features from RISC but also incorporates certain features where CISC is advantageous.

Xtensa processors are typically configurable. CPU designers can enable features such as: additional instructions (both predefined and custom), interrupts, coprocessors, memory management, and others. Some of these features affect the ABI and code generated by the compiler.

Standard Xtensa instructions are 24-bit. Code density option may be enabled to add 16-bit instructions. Wider instructions are also possible in some configurations.

Xtensa processors employ Harvard architecture, meaning that they have separate instruction and data buses. Depending on the SoC design, these buses may be connected to separate instruction and data memories, or to a shared memory.

\section{Registers}

\begin{description}[leftmargin=10em,style=nextline]
  \item[\reg{PC}] Program Counter, holds the address of the instruction being executed. PC is not writeable directly. It can be modified as a side effect of calls, jumps, and exceptions. PC is also not directly readable, however Xtensa provides instructions to perform PC-relative loads and jumps, facilitating access to literal values and generation of position-independent code. 
  \item[\reg{\an}] 16 general purpose 32-bit architectural registers.
  \item[\reg{AR[n]}] Physical general purpose registers. In CPU configurations without the window register option, these are the same as the architectural registers \reg{\an}. In CPU configurations where window register option is enabled, there are more  physical registers than architectural registers. The number of physical registers can be 32 or 64. 16 physical registers are mapped to the architectural registers \reg{\an} at a time.
  \item[Special registers] Xtensa processors contain a number of registers used to control the operation of the processor, perform interrupt and exception handling, etc. Only a few special registers are relevant to the code generation by the compiler. Special registers can not be used as operands of ALU and branch instructions. They must be read and written using \insn{RSR} and \insn{WSR} instructions.
  \item[\reg{SAR}] Shift amount register is a special register. It is used to store the number of bits for subsequent shift instructions. Xtensa does not provide shift instructions which would have the shift amount specified in a general register (\reg{\an}) operand.
  \item[User registers] These registers are added by various processor configuration options, or by processor designers defining custom instructions. Only a few special registers are relevant to code generation by the compiler. Like special registers, user registers can not be used as operands of ALU and branch instructions. They must be read and written using \insn{RUR} and \insn{WUR} instructions.
  \item[\reg{THREADPTR}] Thread pointer register is a user register. The system software typically writes a pointer to the TCB of the executing thread into this register. The register is used by the compiler when accessing thread-local variables.
\end{description}

\section{Windowed Register}

\underline{General purpose registers} (GPR) are used to store data temporarily for CPU while performing various operations. These registers are blazing fast but are limited in number (8 -- 32).

Typically, the number of registers present in the register file are equal to the registers directly accessible by the core. The Xtensa core can only access 16 GPR, namely \reg{a0} -- \reg{a15}. So the register file contains 16 registers.

Xtensa also has a Windowed register option, which when enabled, extends this register file to contain 64 registers. Essentially, the register frame (\reg{a0} -- \reg{a15}) acts as a window, through which only 16 registers are visible, that slides on this large register file having 64 registers. And hence the name: Windowed register.

Which 16 registers are visible is controlled by the WindowBase register. WindowBase register indicates where the window starts in the register file. Also, the shifting/rotation of this window occurs in units of 4. That means, the window starts at (WindowBase x 4)$^{th}$ position in the register file.

\begin{figure}[p]
    \center
    \includegraphics[width=0.9\textwidth]{Windowed_register.png}
    \caption{Register window}
    \label{fig:register-window}
\end{figure}

\section{Calling convention}

Xtensa supports two different application binary interfaces (ABI) which also includes the calling conventions.

1. Windowed register ABI

2. Call0 ABI

We will cover only Windowed register ABI.

\subsection{Windowed register calling convention}

Return address is stored in \reg{a0} and the stack pointer is store in \reg{a1}

Arguments to the functions are passed in both, registers and memory (stack). The first six arguments are passed in the registers and remaining go on the stack.

As for return values, they are returned in registers beginning from \reg{a2} till \reg{a5}. If there are more than 4 values to be returned, the caller passes a pointer which is then populated by callee with all the return values.

\begin{longtable}{|p{5cm}|p{5cm}|}
    \hline
    Register & Use \\
    \hline
    \reg{a0} & Return Address\\ \hline
    \reg{a1} & Stack Pointer\\ \hline
    \reg{a2}--\reg{a7} & Incoming Arguments\\ \hline
\end{longtable}

In Xtensa, subroutine calls are initiated using \insn{CALLn} and \insn{CALLXn} instructions, where \insn{n} specifies the amount by which the register window needs to be rotated for the callee. \insn{n} can be equal to 4, 8, or 12.

Note that \insn{CALL0}/\insn{CALLX0} instructions do not follow windowed register calling convention, so further explanation applies for $n \neq 0$.

What does ``rotation of window for the callee'' exactly mean?

When a subroutine is called using \insn{CALLn}/\insn{CALLXn}, WindowBase register is incremented by $n/4$, so the registers visible by callee are different from those visible by the caller because the register window (\reg{a0} -- \reg{a15}) has moved.

In general, for a windowed register call \insn{CALLn}/\insn{CALLXn}:
\begin{itemize}
    \item \reg{a}$_{n}$ of caller will be \reg{a0} of callee
    \item \reg{a}$_{n+1}$ of caller will be \reg{a1} of callee and so on.
\end{itemize}

So the caller needs to put the first argument of the callee in \reg{a}$_{n+2}$, second in \reg{a}$_{n+3}$ and so on.

\fixme{Explain how many arguments are passed in registers and on the stack.}

While returning from the callee function using \insn{RETW} instruction, WindowBase register is decremented by $n/4$. This restores the register window of the caller.

Let’s take an example:

\begin{verbatim}
/*
 * void bar(int x, int y);
 *
 * void func(void)
 * {
 *     ...
 *     foo = bar(x, y);
 *     ...
 * }
 */

func:
    ...
    mov         a10, x    // a10 is bar's a2
    mov         a11, y    // a11 is bar's a3
    call8       bar
    mov         foo, a10  // a10 is bar's a2 (return value)
    ...
\end{verbatim}

 When a function calls another function, it does not have to store its own arguments somewhere else to accommodate the arguments for the callee since the arguments of the callee is at a different physical location. The callee function internally will still use \reg{a2} to access its first argument but as you can see, \reg{a2} of the caller is at a different physical location than \reg{a2} of callee. If there was no windowing and the number of physical registers would be exactly 16 then \reg{a2} of caller and callee would be same. Thus for each function call, the data in these registers would have to be stored at some other memory location (stack) before calling any function and restore again after returning.

Accessing any memory location, other than register, is very slow and as a result this saving/restoring will have a negative impact on performance. So using windowed register convention saves us the overhead of such stores/restores and also reduces the code size.

\subsection{Stack Layout}

As mentioned, the stack pointer resides in \reg{a1} register. This stack pointer always points to the bottom of the stack!

Usually, function prologue sets up the stack for a function.

In Xtensa, ENTRY instruction is the function prologue

ENTRY instruction primarily does two things:
1. Allocates the stack frame for the function and sets the stack pointer.
2. Moves/rotates the register window by n as specified in the calln/callxn instruction.

Stack layout is always better explained through an illustration \ref{fig:window-abi-stack-layout}.

\begin{figure}[p]
    \center
    \begin{drawstack}
        \startframe
        \cell{Extra save area ($i-1$)}
        \cell{Locals ($i-1$)}
        \cell{Outgoing arguments}
        \cellptrdiv{\reg{SP}$_{i-1}$}
        \cell{Base save area ($i-2$)}
        \finishframe{Stack frame $i-1$ (previous)}
        \startframe
        \padding{0}{}
        \cell{Extra save area ($i$)}
        \cell{Locals ($i$)}
        \cell{Outgoing arguments}
        \cellptrdiv{\reg{SP}$_{i}$}
        \cell{Base save area ($i-1$)}
        \finishframe{Stack frame $i$ (current)}
        \draw[<-,line width=0.7pt] +(3,0) -- +(3,-1);
        \draw (3.1,-0.5) node[anchor=west] {Higher address};
        \draw[<-,line width=0.7pt] +(3,-9) -- +(3,-8);
        \draw (3.1,-8.5) node[anchor=west] {Stack growth};
    \end{drawstack}
    \caption{Windowed ABI stack layout}
    \label{fig:window-abi-stack-layout}
\end{figure}

For clarity, lets use \reg{sp} as stack pointer instead of \reg{a1}.

Like most architectures, in Xtensa too, stack grows downwards. If there are outgoing arguments, apart from the first 6 arguments, then they will go on the positive offset from \reg{sp}. i.e 7th argument on \reg{sp}, 8th on \reg{sp} + 4 and so on. Above the outgoing arguments, local variables of that function are stored.

The region underneath the stack pointer, called Base Save Area, is of 16 bytes and reserved for saving the \reg{a0} -- \reg{a3} of the caller (previous frame) when the window overflow exception occurs. If more registers of the caller are required to be saved then it is stored in the Extra Save Area at the top of the caller (previous) stack frame. The location of saving registers of the caller (i-1) frame is highlighted in the image.

With all the necessary points covered, let’s take an example and connect all the dots.

Suppose, each function call is carried out using call8 and we start with WindowBase = 4

Function A calls B, B calls C, C calls D... till I, i.e:

\newcommand{\calls}{\textrightarrow{}}
\begin{longtable}{lc}
Functions&  A \calls B \calls C \calls D \calls E \calls F \calls G \calls H \calls I\\
WindowBase& 4 \calls 6 \calls 8 \calls 10 \calls 12 \calls 14 \calls 0 \calls 2 \calls 4\\
\end{longtable}
\let\calls\undefined

On each function call, the WindowBase will be incremented by 2 because call8 is used.

No. of bits in WindowBase register = $log_{2}$((No. of registers in register file)/4) = $log_{2}$(64⁄4) = 4. Thus the max value of WindowBase is 15.

As we have noticed, on the 9th function call the window wraps around to a point where the frame contains the data of a parent function, i.e \reg{a0}, \reg{a1}.. contains data of A. It implies that \reg{a8}, \reg{a9}.. of H are \reg{a0}, \reg{a1}.. of A.

A window overflow exception will be generated when H tries to modify \reg{a8}, \reg{a9}.. since it originally contains the context of A, so these must be saved to accommodate arguments of I. At this point, in the window overflow exception handler we must rotate the register window to frame A (WindowBase = 4).

\reg{a0} -- \reg{a3} are stored in the Base Save Area of B’s stack frame. B’s stack frame is accessible since \reg{a9} is \reg{a1} of B, which is B’s stack pointer.
\reg{a4} -- \reg{a7} are stored in the Extra Save Area of A’s stack frame.
Now whenever B returns, window underflow exception will be generated and we need to make sure that the corresponding exception handler would restore these values back into the registers.

